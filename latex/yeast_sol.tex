%% LyX 2.3.8 created this file.  For more info, see http://www.lyx.org/.
%% Do not edit unless you really know what you are doing.
\documentclass[english]{article}
\usepackage[T1]{fontenc}
\usepackage[latin9]{inputenc}
\usepackage{babel}
\usepackage{amsmath}
\usepackage{graphicx}
\usepackage{esint}
\usepackage[unicode=true,pdfusetitle,
 bookmarks=true,bookmarksnumbered=false,bookmarksopen=false,
 breaklinks=false,pdfborder={0 0 1},backref=false,colorlinks=false]
 {hyperref}

\makeatletter

%%%%%%%%%%%%%%%%%%%%%%%%%%%%%% LyX specific LaTeX commands.
%% A simple dot to overcome graphicx limitations
\newcommand{\lyxdot}{.}


\makeatother

\begin{document}
\title{\textbf{Yeast Signaling}}

\maketitle
We begin with the equation:

\begin{align}
\frac{\partial C(x,z)}{\partial t} & =D\nabla^{2}C(x,z)+r\rho\Theta(-x)\delta(z)-\Omega\delta(z)C(x,z)
\end{align}

Look for steady state solutions:

\begin{align}
D\nabla^{2}C(x,z)+r\rho\Theta(-x)\delta(z)-\Omega\delta(z)C(x,z) & =0
\end{align}

Define new coefficients:

\begin{align}
\partial_{xx}C(x,z)+\partial_{zz}C(x,z)+\tilde{r}\Theta(-x)\delta(z)-\tilde{\Omega}\delta(z)C(x,z) & =0
\end{align}

And drop the tildes from here forward.

What happens if we integrate over an $\epsilon$ region around $z=0$:

\begin{align}
\partial_{z}C(x,z)|_{z=+\epsilon}-\partial_{z}C(x,z)|_{z=-\epsilon} & =-r\Theta(-x)+\Omega C(x,z=0)
\end{align}

This is the virtual flux of $\text{\ensuremath{\alpha}}$ factors
through the $z$-plane, and we can turn it into a Robin BC:

\begin{align*}
\partial_{z}C(x,z)|_{z=0} & =-r\Theta(-x)+\Omega C(x,z=0)
\end{align*}

And just solve the Poisson equation on the positive half volume!

Assume structure of $C(x,z)$:

\begin{align}
C(x,z) & =X(x)Z(z)+\frac{r}{2\Omega}
\end{align}

We assume this structure due to the fact that one can deduce from
the equation that:

\begin{equation}
C(-x,z)+C(x,z)=\frac{r}{\Omega}
\end{equation}

\begin{equation}
\Rightarrow C(0,z)=\frac{r}{2\Omega}
\end{equation}

And also:

\begin{equation}
C(x,z)=X(x)Z(z)\Rightarrow C(0,z)=X(0)Z(z)
\end{equation}

Where $X(0)$ is a constant, and there is no way to satisfy the previous
condition for $C(0,z)$ besides constant $Z(z)$, which cannot be
a solution.

So we have BC:

\begin{equation}
\partial_{z}C(x,z)|_{z=L_{z}}=0
\end{equation}

\begin{equation}
\partial_{z}C(x,z)|_{z=0}=-r\Theta(-x)+\Omega C(x,z=0)
\end{equation}

\begin{equation}
C(x,z)|_{x=0}=\frac{r}{2\Omega}
\end{equation}

\begin{equation}
\partial_{x}C(x,z)|_{x=-L_{x}}=0
\end{equation}

And we solve for the $x<0$ region, which dictates the $x>0$ region.

This is the simplest structure that has the potential to adhere to
the constraints, where $C(x,z)$ is still seperable in the Poisson
equation. 

Before the analytical solution, we perform a coarse-grained simulation
of the above to see how results should look like (with finite $L_{x},L_{z}$)
in $2\text{D}$ (\ref{fig: Figure 1}), and plot $C(x,z=0)$ for different
values of $\Omega$ (fig. 2).

\begin{figure}
\begin{centering}
\includegraphics[scale=0.5]{\string"/Users/yotamlifschytz/Desktop/Masters Misc./python/Random/Yeast/figs/omega_0.6\string".png}
\par\end{centering}
\caption{Heatmap of the $C(x,z)$ steady state (simulation taken to $10^{6}$
steps); Arrows represent the flow field at every point; Parameter
values are $r_{0}=0.1$, $L_{x}=20$, $L_{z}=30$.\label{fig: Figure 1}}
\end{figure}

\begin{figure}
\begin{centering}
\includegraphics[scale=0.5]{\string"/Users/yotamlifschytz/Desktop/Masters Misc./python/Random/Yeast/figs/z0_line_plots_combined\string".png}
\par\end{centering}
\caption{Simulation results for $C(x,z=0)$ for different $\Omega$ values\label{fig: 2}}
\end{figure}

Now solve:

\begin{align}
\nabla^{2}C(x,z) & =0
\end{align}

\begin{align}
\frac{1}{X(x)}\partial_{xx}X(x) & =-k^{2}=-\frac{1}{Z}\partial_{zz}Z(z)
\end{align}

\begin{align}
X_{n}(x) & =A_{n}e^{+ik_{n}x}+B_{n}e^{-ik_{n}x}
\end{align}

\begin{align}
Z_{n}(z) & =C_{n}e^{+k_{n}z}+D_{n}e^{-k_{n}z}
\end{align}

BC 3:

\begin{align}
C(x,z)|_{x=0} & \propto A_{n}+B_{n}+\frac{r}{2\Omega}=\frac{r}{2\Omega}
\end{align}

\begin{equation}
\Rightarrow B_{n}=-A_{n}
\end{equation}

\begin{align}
 & \Rightarrow X_{n}(x)=A_{n}sin(k_{n}x)
\end{align}

BC 4:

\begin{align}
\partial_{x}C(x,z)|_{x=-L_{x}} & \propto cos(k_{n}L_{x})=0
\end{align}

\begin{align}
k_{n} & =\frac{\pi}{2L_{x}}(2n+1)\Rightarrow k_{n_{odd}}=\frac{\pi}{2L_{x}}n_{odd}
\end{align}

Where we have only odd wavenumbers.

BC 1:

\begin{align}
\partial_{z}C(x,z)|_{z=L_{z}} & \propto C_{n}e^{+k_{n}L_{z}}-D_{n}e^{-k_{n}L_{z}}=0
\end{align}

\begin{align}
D_{n} & =C_{n}e^{+2k_{n}L_{x}}
\end{align}

\begin{align}
 & \Rightarrow Z_{n}(z)\propto(e^{+k_{n}z}+e^{+2k_{n}L_{x}}e^{-k_{n}z})\propto cosh(k_{n}(z-L_{z}))
\end{align}

So we now have for $x<0$:

\[
C(x,z)=\sum_{n_{odd}}A_{n}sin(k_{n}x)cosh(k_{n}(z-L_{z}))
\]

And our final BC is (on $x<0$):

\begin{align}
\partial_{z}C(x,z)|_{z=0} & =-r+\Omega C(x,z=0)
\end{align}

\begin{align}
\sum-k_{n}A_{n}cos(k_{n}x)sinh(k_{n}L_{z}) & =+\tilde{\Omega}\sum A_{n}sin(k_{n}x)cosh(k_{n}L_{z})-\frac{r}{2}
\end{align}

\begin{align}
\sum_{n_{odd}}A_{n}sin(k_{n}x)[\Omega cosh(k_{n}L_{z})+k_{n}sinh(k_{n}L_{z})] & =\frac{r}{2}
\end{align}

\begin{align}
A_{n}^{'} & :=A_{n}\frac{\Omega cosh(k_{n}L_{z})+k_{n}sinh(k_{n}L_{z})}{r/2}
\end{align}

\begin{align}
\sum_{n_{odd}}A_{n}^{'}sin(k_{n}x) & =1
\end{align}

Multiply by $sin(k_{m}x)$ and integrate on $x<0$:

\begin{align}
A_{m}^{'}\frac{L_{x}}{2} & =\int_{-L_{x}}^{0}sin(k_{m}x)dx=-\frac{1}{k_{m}}
\end{align}

Where we get that for $m_{odd}\neq n_{odd}$ these sines are orthogonal
on $[-L_{x},0].$

\begin{equation}
-\frac{2}{k_{n}L_{x}}=A_{n}^{'}=A_{n}\frac{\Omega cosh(k_{n}L_{z})+k_{n}sinh(k_{n}L_{z})}{r/2}
\end{equation}

\begin{equation}
\Rightarrow A_{n}=-\frac{r}{L_{x}}\frac{1}{k_{n}(\Omega cosh(k_{n}L_{z})+k_{n}sinh(k_{n}L_{z}))}
\end{equation}

\begin{equation}
C_{<}(x,z)=\frac{r}{2\Omega}-\frac{r}{L_{x}}\sum_{n_{odd}}\frac{1}{k_{n}(\Omega cosh(k_{n}L_{z})+k_{n}sinh(k_{n}L_{z}))}sin(k_{n}x)cosh(k_{n}(z-L_{z}))
\end{equation}

And this actually agrees with the symmetry we found, so this is the
solution on all $x$: 

\begin{equation}
C(x,z)=\frac{r}{2\Omega}-\frac{r}{L_{x}}\sum_{n_{odd}}\frac{1}{k_{n}(\Omega cosh(k_{n}L_{z})+k_{n}sinh(k_{n}L_{z}))}sin(k_{n}x)cosh(k_{n}(z-L_{z}))
\end{equation}

Using some hyperbolic identities:

\begin{equation}
C(x,z)=\frac{r}{2\Omega}-\frac{r}{L_{x}}\sum_{n_{odd}}\frac{1}{k_{n}(\Omega+k_{n}tanh(k_{n}L_{z}))}sin(k_{n}x)[cosh(k_{n}z)-sinh(k_{n}z)tanh(k_{n}L_{z})]
\end{equation}

Grapic representation of this analytical result can be found in (\ref{fig: 3}).

\begin{figure}
\begin{centering}
\includegraphics[width=10cm]{\string"/Users/yotamlifschytz/Desktop/Masters Misc./python/Random/Yeast/figs_analytic/omega_0.6\string".png}
\par\end{centering}
\caption{Plotted is the analytical result derived for $C(x,z)$ for the case
of finite $L_{x},L_{z}$ ; Arrows represent the flow field at every
point; Parameter values are $L_{x}=20,L_{z}=40,\Omega=0.6,r=0.1,N_{max}=100$
where the last term represents the cutoff term for the series solution.
\label{fig: 3}}

\end{figure}

Now, we are most interested at what happens on $z=0$ plane, so:

\begin{align}
C(x,0) & =\frac{r}{2\Omega}-\frac{r}{L_{x}}\sum_{n_{odd}}\frac{1}{k_{n}(\Omega cosh(k_{n}L_{z})+k_{n}sinh(k_{n}L_{z}))}sin(k_{n}x)cosh(k_{n}L_{z})
\end{align}

\begin{align}
=\frac{r}{2\Omega}-\frac{r}{L_{x}}\sum_{n_{odd}}\frac{1}{k_{n}(\Omega+k_{n}tanh(k_{n}L_{z}))}sin(k_{n}x)
\end{align}

And at $L_{z}\gg1$:

\begin{align}
\approx\frac{r}{2\Omega}-\frac{r}{L_{x}}\sum_{n_{odd}}\frac{1}{k_{n}(\Omega+k_{n})}sin(k_{n}x)
\end{align}

Which is exactly the same solution as with absorbing BC at $L_{z}=\infty$.

We can further simplify this solution:

\begin{equation}
\frac{r}{2\Omega}-\frac{r}{L_{x}}\sum_{n_{odd}}\frac{1}{k_{n}(\Omega+k_{n})}sin(k_{n}x)=
\end{equation}

\begin{equation}
=\frac{r}{2\Omega}-\frac{r}{\Omega L_{x}}\sum_{n_{odd}}(\frac{1}{k_{n}}-\frac{1}{\Omega+k_{n}})sin(k_{n}x)
\end{equation}

\begin{equation}
=\frac{r}{2\Omega}-\frac{r}{2\Omega}\sum_{n_{odd}}4\frac{sin(k_{n}x)}{\pi n}+\frac{r}{\Omega L_{x}}\sum_{n_{odd}}\frac{sin(k_{n}x)}{k_{n}+\Omega}
\end{equation}

We know $\sum_{n_{odd}}4\frac{sin(k_{n}x)}{\pi n}$ is the fourier
transform of a sin square wave, so:

\begin{equation}
=\frac{r}{\Omega}(\Theta(-x)+\frac{1}{L_{x}}\sum_{n_{odd}}\frac{sin(k_{n}x)}{k_{n}+\Omega})
\end{equation}

We plot this solution

\begin{figure}
\begin{centering}
\includegraphics[scale=0.4]{\string"/Users/yotamlifschytz/Desktop/Masters Misc./python/Random/Yeast/figs_analytic/vary_Omega_plots_Lx_20\string".png}
\par\end{centering}
\caption{Analytical results for $C(x,0)$ for different values of $\Omega$;
Parameter values are $L_{x}=20,r=0.1,N_{max}=10^{5}$}

\end{figure}

Now we can return to original notation of $r,\Omega:$

\begin{equation}
=\frac{r\rho}{\Omega}(\Theta(-x)+\frac{1}{L_{x}}\sum_{n_{odd}}\frac{sin(k_{n}x)}{k_{n}+\frac{\Omega}{D}})
\end{equation}

Take to the continuum limit:

\[
y=k_{n}=\frac{\pi n}{2L_{x}};dy=\frac{\pi}{2L_{x}}dn;dn=2
\]

\[
\frac{1}{L_{x}}\sum_{n_{odd}}\frac{sin(k_{n}x)}{k_{n}+\frac{\Omega}{D}}=\frac{1}{\pi}\intop_{0}^{\infty}\frac{sin(yx)}{y+\frac{\Omega}{D}}dy=\frac{1}{\pi}\intop_{\frac{\Omega}{D}}^{\infty}\frac{sin(yx-\frac{\Omega}{D}x)}{y}dy
\]

\[
=\frac{1}{\pi}\intop_{\frac{\Omega}{D}}^{\infty}\frac{sin(yx)cos(\frac{\Omega}{D}x)-cos(yx)sin(\frac{\Omega}{D}x)}{y}dy=\frac{cos(\frac{\Omega}{D}x)}{\pi}\intop_{\frac{\Omega}{D}}^{\infty}\frac{sin(yx)}{y}dy-\frac{sin(\frac{\Omega}{D}x)}{\pi}\intop_{\frac{\Omega}{D}}^{\infty}\frac{cos(yx)}{y}dy
\]

\[
=\frac{cos(\frac{\Omega}{D}x)}{\pi}[\frac{\pi}{2}sign(x)-Si(\frac{\Omega}{D}x)]-\frac{sin(\frac{\Omega}{D}x)}{\pi}[\frac{\pi}{2}sign(x)-Ci(\frac{\Omega}{D})]
\]

\[
=
\]

\end{document}
