\documentclass{article}
\usepackage{amsmath, amssymb, mathtools, mhchem, stackrel, stmaryrd}

\title{\textbf{Yeast Signaling}}
\date{}

\begin{document}
\maketitle

We begin with the equation:
\begin{align}
\frac{\partial C(x,z)}{\partial t} &= D \nabla^{2} C(x,z) + r \Theta(-x) \delta(z) - \Omega \delta(z) C(x,z)
\end{align}

Look for steady-state solutions:
\begin{align}
D \nabla^{2} C(x,z) + r \Theta(-x) \delta(z) - \Omega \delta(z) C(x,z) &= 0
\end{align}

Define new coefficients:
\begin{align}
\partial_{xx} C(x,z) + \partial_{zz} C(x,z) + \tilde{r} \Theta(-x) \delta(z) - \tilde{\Omega} \delta(z) C(x,z) &= 0
\end{align}

And drop the tildes from here forward.

What happens if we integrate over an $\epsilon$ region around $z=0$:
\begin{align}
\partial_{z} C(x,z)|_{z=+\epsilon} - \partial_{z} C(x,z)|_{z=-\epsilon} &= -r \Theta(-x) + \Omega C(x,z=0)
\end{align}

This is the virtual flux of $\alpha$ factors through the $z$-plane, and we can turn it into a Robin BC:
\begin{align*}
\partial_{z} C(x,z)|_{z=0} &= -r \Theta(-x) + \Omega C(x,z=0)
\end{align*}

And just solve the Laplace equation on the positive half volume.

Assume structure of $C(x,z)$:
\begin{align}
C(x,z) &= X(x) Z(z) + \frac{r}{2 \Omega}
\end{align}

We assume this structure due to the fact that one can deduce the following symmetry from the equation / BC:
\begin{equation}
C(-x,z) + C(x,z) = \frac{r}{\Omega}
\end{equation}

Thus:
\begin{equation}
C(0,z) = \frac{r}{2 \Omega}
\end{equation}

And also:
\begin{equation}
C(x,z) = X(x) Z(z) \Rightarrow C(0,z) = X(0) Z(z)
\end{equation}

Where $X(0)$ is a constant, and there is no way to satisfy the previous condition for $C(0,z)$ besides constant $Z(z)$, which cannot be a solution.

The symmetry can be seen as follows - if $C(x,z)$ is a solution, for $x>0$ we have:
\[
\partial_{z} C(x,z)|_{z=0} = \Omega C(x,z=0)
\]

And:
\[
\partial_{z} C(-x,z)|_{z=0} = -r + \Omega C(-x,z=0)
\]

Now take the function $\frac{r}{\Omega} - C(x,z)$; it satisfies the Robin BC for $x<0$ as well:
\[
-\partial_{z} C(x,z)|_{z=0} = \Omega \left(\frac{r}{\Omega} - C(x,z)\right) = r - \Omega C(x,z=0)
\]

It also satisfies the rest of the reflecting BC and is a solution of the Laplace equation as well, thus $\frac{r}{\Omega} - C(x,z)$ is also a solution on $x<0$.

From uniqueness, we can deduce $\frac{r}{\Omega} - C(x,z) = C(-x,z)$ and we get our symmetry.

\textbf{Boundary Conditions:}
\begin{align}
\partial_{z} C(x,z)|_{z=L_{z}} &= 0 \\
\partial_{z} C(x,z)|_{z=0} &= -r \Theta(-x) + \Omega C(x,z=0) \\
C(x,z)|_{x=0} &= \frac{r}{2 \Omega} \\
\partial_{x} C(x,z)|_{x=-L_{x}} &= 0
\end{align}

We solve for the $x<0$ region, which dictates the $x>0$ region.

\textbf{Laplace Equation Solution:}
\begin{align}
\nabla^{2} C(x,z) &= 0 \\
\frac{1}{X(x)} \partial_{xx} X(x) &= -k^{2} = -\frac{1}{Z(z)} \partial_{zz} Z(z)
\end{align}

\begin{align}
X_{n}(x) &= A_{n} e^{+ik_{n} x} + B_{n} e^{-ik_{n} x} \\
Z_{n}(z) &= C_{n} e^{+k_{n} z} + D_{n} e^{-k_{n} z}
\end{align}

**Full solution and further details continue here,** following the pattern above.

\end{document}
